\chapter*{Resumo}
\thispagestyle{empty}

Neste trabalho descrevemos uma abordagem para realizar s�ntese em sistemas de controle discreto com realimenta��o de estados com respeito a requisitos de performance a qual � baseada em t�cnicas de s�ntese indutiva guiada por contraexemplo (CEGIS). Nessa abordagem verifica-se um requisito de performance (\textit{e.g.} tempo de assentamento e m�ximo sobressinal) em um determinado sistema de controle com o objetivo de saber se satisfaz ao valor desejado para esse requisito, caso n�o satisfa�a, deve-se encontrar um sistema que satisfa�a ao requisito desejado, e nesse caso, redefine-se o controlador do sistema. Para gera��o do controlador, utilizamos uma t�cnica de aprendizagem onde a cada itera��o que a verifica��o do requisito n�o satisfa�a, aprende-se que esse controlador n�o nos serve. Na verifica��o desses requisitos de performance em sistemas de controle discreto, consideramos a fragilidade (\textit{erro de quantiza��o num�rica, arredondamento, efeitos de FWL, etc}) nos controladores utilizados. Essa abordagem � �til para auxiliar engenheiros de controle em seus projetos de sistemas de controle discreto, visto que essas fragilidades ocorrem durante a implementa��o em uma plataforma digital, e nesse caso, essa abordagem gera o sistema que satisfaz os requisitos desejados no projeto. Essa abordagem foi implementada dentro do DSVerifier que � uma ferramenta que emprega verifica��o de modelo limitada (e ilimitada) baseada em, teorias de m�dulo de satisfatibilidade. Nossa abordagem foi avaliada em um conjunto de \textit{benchmarks} cl�ssicos de sistemas de controle extra�dos da literatura, bem como em \textit{benchmarks} espec�ficos considerando diferentes autovalores e em diferentes configura��es. Os resultados experimentais mostram que a abordagem elaborada � eficaz para a s�ntese de requisitos de perfomance em sistemas de controles discreto com realimenta��o de estados, visto que considera problemas pr�ticos de implementa��o (efeitos FWL), diferentemente de outros m�todos que rotineiramente n�o consideram esses problemas.

\vspace*{\stretch{1}}

\noindent \textsf{Palavras-chave:} Sistemas de controle, Verifica��o de Parametros de Performance, Sintese.
%}

\cleardoublepage
