\chapter*{Resumo}
\thispagestyle{empty}

Este trabalho descreve uma nova abordagem para localizar falhas em programas concorrentes, a qual � baseada em t�cnicas de verifica��o de modelos limitada e sequencializa��o. A principal novidade dessa abordagem � a ideia de reproduzir um comportamento defeituoso em uma vers�o sequencial do programa concorrente. De forma a apontar linhas defeituosas, analizam-se os contraexemplos gerados por um verificador de modelos para o programa sequencial instrumentado e procura-se um valor para uma vari�vel de diagn�stico, o qual corresponde a linhas reais no programa original. Essa abordagem � �til para aperfei�oar o processo de depura��o para programas concorrentes, j� que ela diz qual linha deve ser corrigida e quais valores levam a uma execu��o bem-sucedida. Essa abordagem foi implementada como uma transforma��o c�digo-a-c�digo de um programa concorrente para um n�o-determin�stico sequencial, o qual � ent�o usado como entrada para ferramentas de verifica��o existentes. Resultados experimentais mostram que a abordagem descrita � eficaz e � capaz de localizar falhas na maioria dos casos de teste utilizados, extra�dos da su�te da {\it International Competition on Software Verification} $2015$.

\vspace*{\stretch{1}}

\noindent \textsf{Palavras-chave:} Verifica��o de Modelos, Localiza��o de Falhas, Sequencializa��o, Depura��o.
%}

\cleardoublepage
